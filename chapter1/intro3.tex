近年 Apple の Siri, NTT の しゃべってコンシェルなど,人と機械が対話することによってユーザの要求を達成する対話システムが一般に普及している.しかし,そのような対話システムが行う対話は一問一答型の対話がほとんどであり,ユーザと応答を繰り返してユーザの特定の目標を達成するという対話はできない.これはシステムが,ユーザが過去の発話で発した要求などといった情報を保持していないからである.ゆえに,システムがユーザとの対話の流れを理解するための技術である対話状態追跡(Dialogue State Tracking)が必要であり,その研究が行われている.\par
対話システムにおける対話状態追跡は,対話履歴を考慮して対話状態を維持あるいは更新する役割を担う.対話状態はシステムが対話中のユーザの目標や要求を保持するための枠組みである.システムはこの対話状態によってユーザが過去の発話で発した要求などを理解することが可能となる.対話システムにおいて,対話状態追跡は言語理解の後に行われる.対話システムにおける言語理解は,ユーザの発話を入力として,ユーザの目標と,スロットと値の組で表現される要求を出力する.そして対話状態追跡は,その出力結果と前の対話状態を入力として,更新した対話状態を出力する.\par
従来の対話状態追跡はルールベースで行われていたが,現在はニューラルネットワークなどの機械学習で対話状態追跡を行う研究が盛んに行われている.中でも,深層学習を用いた End-to-End 型のモデル\cite{nbt,e2e}が優れた結果を残している.End-to-End 型の対話状態追跡モデルは,言語理解の出力結果の代わりに発話文そのものを入力として,対話状態を出力する.従来法では過去の複数ターンの発話文を単純に入力としている.ターンとは対話の順序を示すもので,各ターンはユーザあるいはシステムの発話を持つ.従来法のように過去の複数ターンの発話文を入力すると,“Yes” や “No” など対話状態追跡に不要な文を含む可能性がある.そこで本研究では,発話の意図を示す対話行為タグを用いて,長い対話履歴の中から特に対話を決めるのに重要だと思われる発話を抽出する手法を提案する.\par
また,対話状態追跡のコンペティションである  Dialog System Technology Challenges 8 Track4 Dialogue State Tracking (DSTC8-Track4)\cite{dstc8} が開催されている.本研究では DSTC8-Track4 のタスク設定に基づいて評価を行う.\par
以下,第 2 章では近年の対話システムにおける対話状態追跡とその問題点について,第 3 章では本研究におけるベースラインとなるモデルについて述べる.また,第 4 章では提案手法について述べる.
第 5 章では対話行為の重要性について調べた予備実験の結果を示し,第 6 章では提案手法の結果と分析を示す.第 7 章では,むすびと今後の課題について述べる.
