近年 Apple の Siri, Amazon の Alexa, Google の Google Assistant など,人と機械が対話することによってユーザの要求を達成するタスク指向型対話システムが一般に普及している.タスク指向型対話システムは,今後更に利用可能なドメインや機能が増加していくことが期待される.利用可能なドメインや機能が増加すれば,ユーザの発話パターンが増えて,システムがユーザの目標や要求を的確に理解するのが困難になると考えられる.ゆえに,対話システムにおいてシステムがユーザの目標や要求を推定するために使用される対話状態追跡(Dialogue State Tracking)の研究が行われている.\par
対話状態追跡は対話全体を考慮した言語理解と定義されている.言語理解とは,各ターンでのユーザの発話文から,ユーザが達成しようとしている目標と,スロットと値のペアとして表される要求を推定し,システムが理解できる形にすることである.通常,言語理解は各ターンの発話のみを考慮するため,対話全体を考慮できない.よって,対話システムでは,DST が言語理解の結果と対話全体を考慮して,ユーザの目標と要求を保持する枠組みである対話状態を維持あるいは更新する.そして,対話状態から適切な応答文を生成してユーザに返すというのが一般的な対話システムである.\par
従来の対話状態追跡はルールベースで行われていたが,現在はニューラルネットワークなどの機械学習で対話状態追跡を行う研究が盛んに行われている.中でも,深層学習を用いた End-to-End 型のモデル\cite{nbt,e2e}が優れた結果を残している.一方,対話状態追跡モデルのほとんどは現在のターンでのユーザ発話と前のターンでのシステムの対話行為を入力として推定を行う.しかし,多くの場合,考慮されたユーザ発話やシステムの対話行為は十分な情報を与えない.ゆえに,過去の発話を参照するために対話履歴を用いる必要があるが,単純に複数ターンの発話を加えると,スロット値推定に必要な情報を持たない発話 “Yes” や “No” などを入力に含んでしまいノイズとなる.そこで本研究では,長い対話履歴の中から特に対話を決めるのに重要だと思われる発話を抽出するために,対話行為を用いて抽出する手法を提案する.\par
また,対話状態追跡のコンペティションである  Dialog System Technology Challenges 8 Track4 Dialogue State Tracking (DSTC8-Track4)\cite{dstc8} が開催されている.本研究では DSTC8-Track4 のタスク設定に基づいて評価を行う.\par
以下,第 2 章では近年の対話システムにおける対話状態追跡とその問題点について,第 3 章では本研究におけるベースラインとなるモデルについて述べる.また,第 4 章では提案手法について述べる.
第 5 章では対話行為の重要性について調べた予備実験の結果を示し,第 6 章では提案手法の結果と分析を示す.第 7 章では,むすびと今後の課題について述べる.
