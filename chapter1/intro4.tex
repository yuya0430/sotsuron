対話システムは,ユーザが持つ要求を対話によって達成することを目的とするタスク指向型対話システムと,対話そのものを目的とする雑談対話システムに大別される.タスク指向型対話システムの例として Apple の「Siri」, NTT の「しゃべってコンシェル」などがあるが,その多くが行う対話は一問一答型である.一問一答型は簡単に実用化できるため多くのシステムで用いられているが,ユーザと応答を繰り返してユーザの細かな要求に対応するという対話はできない.これはシステムが,ユーザが過去の発話で発した要求などといったこれまでの対話の流れを理解していないからである.ゆえに,システムがユーザとの対話の流れを理解するための技術である対話状態追跡(Dialogue State Tracking)が必要とされている.\par
一問一答型ではないタスク指向型対話システムでは,対話状態と呼ばれる対話中のユーザの目的や要求を保持するための枠組みを用いて対話の流れを理解する.対話システムにおける対話状態追跡は,ユーザが発話する度に対話履歴を考慮してその対話状態を維持あるいは更新する役割を担う.そしてシステムは,更新された対話状態を見て次にユーザに対して行う行動や発話を決定することで,ユーザの目的達成に向かった対話を行う.対話状態追跡を行わない場合,システムが過去の対話で得た情報を保持できないのに加え,次の行動を選択できない.そのため,対話状態追跡はタスク指向型対話システムにとって重要な要素である.
\par
現在の対話状態追跡はニューラルネットワークなどの機械学習に基づく手法が盛んに行われている.中でも,深層学習を用いた End-to-End 型のモデル\cite{nbt,e2e}が優れた結果を残しており,発話文から直接対話状態を出力することが可能となった.このようなモデルでは,対話の流れを捉えるために対話履歴を入力に用いる.
\par
対話履歴を全て入力することで対話の流れを捉えられるが,現在の発話との依存関係がない発話が含まれる上に計算量が多くかかる.そのため,従来研究では対話履歴として直近の数発話を入力とする場合が多い.しかし,直近の数発話では対話状態の推定に貢献しない不要な発話が含まれ,必要な発話が含まれない可能性がある.
\par
そこで本研究では,発話者の意図を示す対話行為タグを用いて,対話履歴の中から対話状態の推定に貢献すると思われる発話を抽出する手法を提案する.対話行為タグによって各発話を意図ごとに分類し,特に状態に影響を与える発話を選択的に抜き出すことで,計算量を抑えつつ長い対話履歴を考慮することが可能であると期待できる.
\par
対話システムの国際コンペティションである  Dialog System Technology Challenges 8 Track4 Dialogue State Tracking (DSTC8-Track4)\cite{dstc8} において,対話状態追跡のデータセットとベースラインモデルが公開されている.本研究では DSTC8-Track4 のデータセットとモデルを用いて実験を行い従来法との比較を行う.\par
以下,第 2 章では近年の対話システムにおける対話状態追跡とその問題点について,第 3 章では本研究におけるベースラインモデルについて述べる.また,第 4 章では提案手法について述べる.
第 5 章ではどの対話行為タグを持つ発話が重要なのかを調査した予備実験の結果を示し,第 6 章では提案手法の結果と分析,さらに従来手法との比較を示す.第 7 章では,むすびと今後の課題について述べる.
