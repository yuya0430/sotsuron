本研究では,対話状態追跡における対話履歴の使い方として,対話行為タグを用いて重要な対話履歴を抽出する手法を提案した.DSTC8-Track4 で提供されたベースラインモデルと比較した結果,ユーザにスロット値候補を提案する “OFFER” という対話行為タグを用いたもので Joint Goal Accuracy が 7.5\% 向上した.直近の数発話を履歴として入力する従来手法と “OFFER” タグを用いた提案手法とを比較した結果,提案手法がモデルの学習時間を減らすのと同時に,性能を向上させることが可能であることを示した.
\par
今回の実験では“OFFER”を持つ発話のみを抽出した場合が最高の結果を示したが,他の対話行為と組み合わせて複数の対話行為を用いることで更なる性能の向上が期待できる.ゆえに,各対話行為タグが各ドメインにどのように影響するのかの調査を行い,対話行為タグの最適な組み合わせを見つけたい.
\par
今回の手法は人間が選択的にどの対話行為タグを持つ発話に注目するかを決めていた.この手法でも性能は向上するが,ドメインごとに注目すべき発話が異なる場合や他のデータセットの場合で重要な対話行為タグや対話履歴が変化するため対応ができない.そのため,機械側で対話履歴中のどの発話に注目すべきかを推定可能な対話状態追跡モデルを検討したい.