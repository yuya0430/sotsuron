モデルの実装には Tensorflow を用いている.実行環境は,産業技術総合研究所が構築し運用する,人工知能処理向け計算インフラストラクチャであるAI橋渡しクラウド(AI Bridging Cloud Infrastructure; ABCI)\cite{abci}を利用している.ABCIの計算資源はタイプ別に分かれているが,本研究では,CPU(Intel Xeon Gold 6148プロセッサー 2.4 GHz)が 5コア,GPU(NVIDIA Tesla V100 for NVLink 16GiB HBM2)が 1 個,メモリが 240 GiB,ローカルストレージが 180 GBの G.small を用いている.
\par
モデルの設定は,与えられたベースラインモデルとミニバッチサイズ以外全て同じにしている.BERT のハイパーパラメータはBERTの論文\cite{bert}に記載された $\verb|BERT|_{\verb|BASE|}$ の構成をそのまま使用する.つまり,BERTは隠れ層が 12 層,隠れ層のサイズが 768 次元である.学習はミニバッチ学習を行い,ミニバッチサイズを学習時は 32,検証時は 8,テスト時は 8 とする.ただし,従来の対話履歴の使用法との比較の際は,学習時のミニバッチサイズを 24 にして実験を行う.入力発話の最大長は,2発話入力の場合で 80 とし,1 発話増えるたびに 40 増加させる.学習率は 1e-4 とする.誤差関数には,交差エントロピーを使用し,最適化アルゴリズムは重み減衰によって過学習を抑制する Weight Decay という手法を用いた Adam \cite{adam} を使用する.過学習を防ぎモデルの汎化性能を上げるために,dropout(dropout 率は 0.1)を用いる.モデルは 80 epoch の学習を行い,学習を終えた段階のモデルを評価に使用する.
\par
データは Single-domain のみを用いて,学習用として 5,403 対話,検証用として 836 対話を学習に用いる.テスト用のデータはまだ未公開のため,今回は検証用データで評価している.
