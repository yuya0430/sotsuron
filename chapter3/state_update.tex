状態更新モジュールの学習に用いるサンプルは各ユーザターンごとに生成される.各サンプルには,入力文となる現在のターンのユーザ発話,前のターンのシステム発話に加えて,扱うサービスのスキーマ埋め込み表現,各種正解ラベルが含まれる.状態更新モジュールはそのサンプルを用いて,インテント,要求スロット,スロット値を推定するように学習する.
\par
まず初めに状態更新モジュールは各サンプルのユーザ発話と前のシステム発話が BERT に送られ,BERT は発話ペアの埋め込み表現とトークンレベルの埋め込み表現を出力する.その後,インテント推定,要求スロット推定,スロット値推定を共同で学習する.
\par
インテント推定は,現在のターンにおけるインテントを推定する.インテント推定では,まず初めに,発話ペアの埋め込み表現とスキーマの埋め込みに含まれる各インテントの埋め込み表現を連結する.各インテントの埋め込みには,インテント無しを表す NONE インテントの埋め込み表現も含まれる.連結された埋め込み表現は訓練可能な射影を使用してロジット値に射影される.ロジット値(logit)とは,確率 $p$ で起こる事象 A について,A が起こる確率と起こらない確率の比の対数 $\log(p / (1-p))$ である.得られた全てのインテントのロジット値は Softmax 関数で正規化され,サービス中の全てのインテントにわたる確率分布が生成される.そして,確率が最大のインテントがそのターンにおけるインテントであると推定される.
\par
要求スロット推定は,ユーザがスロット値を要求しているスロットを推定する.要求スロット推定では,発話ペアの埋め込み表現と全てのスロット埋め込みを用いて,インテント推定と同様の方法でロジット値を取得する.そして,各ロジット値は Sigmoid 関数を使用して正規化され,その値を各スロットのスコアとする.スロットのスコアが 0.5 を超える場合,そのスロットは要求スロットであると推定される.
\par
モデルは直近の発話 2 つを入力として使用するため,前のユーザターンからの対話状態の差異を推定するように学習する.スロット値は2段階で推定する.まず,スロットの状態として,NONE, DONTCARE, ACTIVE を定義する.NONEはスロット値が変更されないことを示し,DONTCAREはユーザがそのスロットに対して関心がないことを示し,ACTIVEはスロット値が変更されることを示す.スロット状態推定では各スロットに対してどの状態かを推定する.入力は,候補ありスロット状態推定ならば候補ありスロットの埋め込み表現,候補なしスロット状態推定ならば候補なしスロットの埋め込み表現を発話ペアの埋め込み表現と連結する.連結した埋め込み表現から,訓練可能な射影を使用して,NONE, DONTCARE, ACTIVEのロジット値を取る分布を取得する.スロットの状態が NONE と推定された場合,スロット値が変更されない.DONTCAREと推測された場合,スロット値として dontcare が割り当てられ,ACTIVE と推測された場合,次の段階でスロット値が推定され,その値が割り当てられる.
\par
候補ありスロット値推定は候補ありスロットに割り当てられる値を推定する.候補ありスロット推定では,発話ペアの埋め込み表現とスロット候補の埋め込み表現を連結して,インテント推測と同様の方法で各スロット値候補のロジット値を得る.得られた全てのロジット値は,Softmax 関数を使用して正規化され,スロット値候補の確率分布が生成される.そして,確率が最大である値がスロットに割り当てられる.この操作は候補ありスロットごとに行われる.
\par
候補なしスロット値推定は,入力文中の単語からスロット値となる文字列を抜き出す必要があるため,スロット値となる文字列の範囲を推定するように学習する.範囲の推定では,BERT モデルから取得したトークンレベルの埋め込み表現と候補なしスロットの埋め込み表現を連結する.そして,連結した埋め込み表現は訓練可能な射影を使用してロジット値に変換される.得られた全てのトークンのロジット値は,Softmax 関数を使用して正規化され,全てのトークンにわたる確率分布にされる.この分布は,範囲の開始位置である start を推定するように学習する.また,異なる重みセットを使用した同様の手順で,範囲の終了位置である end の分布も推定する.推論中,トークンの番号を $i,j$ として,$\verb|start|[i] + \verb|end|[j] \ (i \leq j)$ という確率の和を最大化する範囲が推定され,その範囲にある文字列がスロットに割り当てられる.
\par
これら7つの推定を共同で学習する.7つの推定では,それぞれで交差エントロピー誤差を計算する.計算した7つの誤差の和をこのモデルの誤差として用いる.
\par
状態更新モジュールの候補なしスロット値推定で推定できる値は,現在のターンのユーザ発話と前のターンのシステム発話に含まれるスロット値のみである.ゆえに,ユーザが過去の対話中に出現したスロット値を暗黙的に参照した場合に,そのスロット値を推定できない.この問題を解決するためには,システムが過去の対話の流れを捉える必要がある.