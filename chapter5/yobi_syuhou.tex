\begin{table}[htb]
    \centering
    \caption{SGDデータセットの学習用と検証用データにおける各対話行為の出現回数}
    \label{tab:action}
    \vspace{3mm}
    \begin{tabular}{|c|r|r|r|} \hline
        & \multicolumn{1}{c|}{Single-domain} & \multicolumn{1}{c|}{Multi-domain} & \multicolumn{1}{c|}{Combined} \\ \hline
        INFORM & 5,735 & 19,798 & 25,533 \\ \hline
        REQUEST & 9,772 & 29,841 & 39,613 \\ \hline
        CONFIRM & 6,417 & 18,213 & 24,630 \\ \hline
        OFFER & 10,276 & 33,887 & 44,163 \\ \hline
        NOTIFY\_SUCCESS & 4,215 & 12,254 & 16,469 \\ \hline
        NOTIFY\_FAILURE & 629 & 2,131 & 2,760 \\ \hline
        INFORM\_COUNT & 4,020 & 12,519 & 16,539 \\ \hline
        OFFER\_INTENT & 2,527 & 8,555 & 11,082 \\ \hline
        REQ\_MORE & 3,655 & 6,890 & 10,545 \\ \hline
        GOODBYE & 6,239 & 12,385 & 18,624 \\ \hline
        全システム発話 & 47,258 & 142,087 & 189,345 \\ \hline
    \end{tabular}
\end{table}

予備実験では SGD データセットの学習用と検証用のデータを全て使用した.対話数は18,624 対話,そのうちシステム発話は 189,345 発話ある.調査する発話は,対話行為タグが与えられるシステム発話に限定し,各発話の対話行為は発話が持つ対話行為タグで決定する.複数種類の対話行為タグを持つ場合は,対話行為も複数にしている.そのようにした場合の各対話行為の出現回数を表\ref{tab:action}に示す.このように各対話行為の出現回数には偏りがあるため,予備実験で示す結果は全て1対話あたりとする.なお,予備実験では,対話行為の“INFORM\_COUNT”と“OFFER\_INTENT”が与えるスロット値はスロット値の更新に関係ないため除外している.
\par
まず,各対話行為のスロット値候補の出現回数を調査した.スロット値候補は,各システム発話に与えられているスロット値範囲リストと対話行為リストを用いて重複を許さずに抽出される.各対話行為の発話からスロット値候補を抽出して,スロット値候補の出現回数をカウントしていく.全てをカウントしたのち,対話数で割ることで,1 対話あたりのスロット値候補の出現回数を各対話行為ごとに求める.

\par
続いて,各対話行為で出現したスロット値候補が実際にスロット値に反映された回数を調査した.まず,現在のターンの対話状態と前のターンの対話状態を比較して,現在のターンで追加されたスロット値を抽出する.そして現在のターンまでに各対話行為で出現したスロット値候補が何個含まれているかをカウントしていく.こちらも,スロット値候補の出現回数と同様に 1 対話あたりの数値を求める.