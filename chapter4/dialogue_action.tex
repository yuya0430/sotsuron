SGD データセットに存在する対話行為タグは 10 種類である.SGD データセットでは,システムの発話にのみ対話行為タグが与えられる.対話行為タグの種類と意味に関しては以下に示す.
\begin{itemize}
    \item INFORM - スロットの値をユーザに通知
    \item REQUEST - ユーザにスロットの値を要求
    \item CONFIRM - ユーザから得たスロットの値を確認
    \item OFFER - ユーザにスロットの値を提案
    \item NOTIFY\_SUCCESS - 目的達成に成功したことをユーザに通知
    \item NOTIFY\_FAILURE - 目的達成に失敗したことをユーザに通知
    \item INFORM\_COUNT - ユーザの要求に合致する対象の個数を通知
    \item OFFER\_INTENT - ユーザに新しいインテントを提案
    \item REQ\_MORE - 他に何かするかをユーザに質問
    \item GOODBYE - 対話を終了
\end{itemize}
